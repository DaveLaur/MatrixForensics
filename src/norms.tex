\chapter{Norms}

\section{General Properties}
Matrix norms satisfy some properties:
\begin{align}
f(\mA)    &\ge 0             \\
f(\mA)    &=   0  \iff \mA=0 \\
f(c\mA)   &=   |c|f(\mA)     \\
f(\mA+\mB)&\le f(\mA)+f(\mB)
\end{align}
Many popular norms also satisfy ``sub-multiplicativity": $f(\mA\mB)\le f(\mA)f(\mB)$.

\section{Matrices}

\subsection{Frobenius norm}
\begin{align}
\norm{\mA}_F &= \sqrt{\trace\mA\mA^H}                           \\
             &= \sqrt{\sum_{i=1}^m \sum_{j=1}^n |\mA_{ij}|^2 }  \\
             &= \sqrt{\sum_{i=1}^m \eig(A^H A)_i }
\end{align}

\subsubsection{Special Properties}
\begin{align}
\norm{\mA\vx}_2         &\le \norm{\mA}_F \norm{\vx}_2~~~\vx\in\sRn \\
\norm{\mA\mB}_F         &\le \norm{\mA}_F \norm{\mB}_F \\
\norm{\mC-\vx\vx^T}_F^2 &= \norm{\mC}_F^2+\norm{\vx}_2^4-2 \vx^T \mC \vx
\end{align}

\subsection{Operator Norms}
For $p=1,2,\infty$ or other values, an operator norm indicates the maximum input-output gain of the matrix.
\begin{equation}
\norm{\mA}_p=\max_{\norm{\vu}_p=1} \norm{\mA\vu}_p
\end{equation}

\begin{align}
\norm{\mA}_1
  &=\max_{\norm{\vu}_1=1} \norm{\mA\vu}_1       \\
  &=\max_{j=1,\ldots,n} \sum_{i=1}^m |\mA_{ij}| \\
  &=\textrm{Largest absolute column sum}
\end{align}

\begin{align}
\norm{\mA}_\infty
  &=\max_{\norm{\vu}_\infty=1} \norm{\mA\vu}_\infty  \\
  &=\max_{j=1,\ldots,m} \sum_{i=1}^n |\mA_{ij}| \\
  &=\textrm{Largest absolute row sum}
\end{align}

\begin{align}
\norm{\mA}_2
  &=\textrm{``spectral norm"}                   \\
  &=\max_{\norm{\vu}_2=1} \norm{\mA\vu}_2       \\
  &=\sqrt{\max(\eig(\mA^T\mA))} \\
  &=\textrm{Square root of largest eigenvalue of~}\mA^T\mA
\end{align}



\subsubsection{Special Properties}
\begin{align}
\norm{\mA\vu}_p &\le \norm{\mA}_p \norm{\vu}_p \\
\norm{\mA\mB}_p &\le \norm{\mA}_p \norm{\mB}_p 
\end{align}

\subsection{Spectral Radius}
Not a proper norm.
\begin{equation}
\rho(\mA)=\textrm{spectral radius}(\mA)=\max_{i=1,\ldots,n} | \eig(\mA)_i |
\end{equation}

\subsubsection{Special Properties}
\begin{align}
\rho(\mA) &\le \norm{\mA}_p \\
\rho(\mA) &\le \min(~\norm{\mA}_1, \norm{\mA}_\infty)
\end{align}


\section{Vectors}

\begin{align}
\norm{\vx}_1      &= \sum_i |\vx_i|           & \textrm{L1-norm\index{L1-norm}} \\
\norm{\vx}_p      &= (\sum_i |\vx_i|^p)^{1/p} & \textrm{P-norm\index{P-norm}}   \\
\norm{\vx}_\infty &= \max_i |\vx_i|           & \textrm{L$\infty$-norm\index{L$\infty$-norm}, L-infinity norm}
\end{align}

\subsection{Identities}

\begin{align}
2\norm{\vu}_2^2+2\norm{\vv}_2^2 &= \norm{\vu+\vv}_2^2 + \norm{\vu-\vv}_2^2                                                      & \textrm{Polarization Identity} \\
<\vx,\vy>                       &= \frac{1}{4}\left(\norm{\vx+\vy}_2^2-\norm{\vx-\vy}_2^2\right)~~\forall \vx,\vy\in\mathcal{V} & \textrm{Polarization Identity} \\
\norm{u}_2^2+\norm{v}_2^2&=\norm{\begin{bmatrix} u \\ v\end{bmatrix}}_2^2
\end{align}


\subsection{Bounds}

\begin{align}
|\vx^T \vy| &\le \norm{\vx}_2 \norm{\vy}_2 & \textrm{Cauchy-Schwartz Inequality} \\
|\vx^T \vy| &\le \sum_{k=1}^n |\vx_k \vy_k| \le \norm{\vx}_p \norm{\vx}_q~~~\forall p,q\ge1: 1/p+1/q=1 & \textrm{H\"older Inequality}
\end{align}

For $\vx\in\mathbb{R}^n$
\begin{equation}
\frac{1}{\sqrt{n}}\norm{\vx}_2
\le\norm{\vx}_\infty
\le\norm{\vx}_2
\le\norm{\vx}_1
\le\sqrt{\textrm{card}(\vx)}\norm{\vx}_2
\le\sqrt{n}\norm{\vx}_2
\le n \norm{\vx}_\infty
\end{equation}

For any $0<p<q$ we have that $\norm{\vx}_q\le\norm{\vx}_p$.